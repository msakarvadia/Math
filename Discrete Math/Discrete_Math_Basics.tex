\documentclass{article}
\usepackage[utf8]{inputenc}
\usepackage{graphics}

\usepackage{fancyhdr}
\usepackage{amssymb}

\pagestyle{fancy}
\fancyhf{}
\lfoot{Mansi Sakarvadia}
\lhead{Discrete Mathematics}
\rfoot{Page \thepage}

\begin{document}

\begin{titlepage} % Suppresses headers and footers on the title page

	\centering % Centre everything on the title page
	
	\scshape % Use small caps for all text on the title page
	
	\vspace*{\baselineskip} % White space at the top of the page
	
	%------------------------------------------------
	%	Title
	%------------------------------------------------
	
	\rule{\textwidth}{1.6pt}\vspace*{-\baselineskip}\vspace*{2pt} % Thick horizontal rule
	\rule{\textwidth}{0.4pt} % Thin horizontal rule
	
	\vspace{0.75\baselineskip} % Whitespace above the title
	
	{\LARGE DISCRETE MATHEMATICS\\ MATH 381\\} % Title
	
	\vspace{0.75\baselineskip} % Whitespace below the title
	
	\rule{\textwidth}{0.4pt}\vspace*{-\baselineskip}\vspace{3.2pt} % Thin horizontal rule
	\rule{\textwidth}{1.6pt} % Thick horizontal rule
	
	\vspace{2\baselineskip} % Whitespace after the title block
	
	%------------------------------------------------
	%	Subtitle
	%------------------------------------------------
	
	Basic concepts and examples explaining the fundamentals of Discrete Mathematics. % Subtitle or further description
	
	\vspace*{3\baselineskip} % Whitespace under the subtitle
	
	%------------------------------------------------
	%	Editor(s)
	%------------------------------------------------
	
	By
	
	\vspace{0.5\baselineskip} % Whitespace before the editors
	
	{\scshape\Large Mansi Sakarvadia} % Editor list
	
	\vspace{0.5\baselineskip} % Whitespace below the editor list
	
	\textit{The University of North Carolina \\ Chapel Hill} % Editor affiliation
	
	
\end{titlepage}

\tableofcontents

\section{February 17, 2020}
   
Recall:
\begin{itemize}
  \item Functions f: A\textrightarrow  B
  \item Image $Im(f) = f(a)$
\end{itemize}

\noindent If f(a) = b, say "a is a preimage of b"
\begin{itemize}
  \item $Gr(f) = \{(a,b) | f(a)=b\}\subseteq AxB$
  \item $Gr(f) = \{(a,f(a)) | a \in A\}$
\end{itemize}

\noindent Graph $Gr(f)$ is a relation between A and B\\
\\
\underline{Which binary relations (subsets of AxB) are graphs of functions?}
\begin{itemize}
  \item A subset $s\subseteq AxB$ is the graph of a function if for every element $a \in A$, there is a unique element $b \in B$ such that $(a,b) \in S$.
  \item \underline{key} Can't have $(a, b_1)$ and $(a, b_2)$ $\in$ S where $b_1 != b_2$ and expect S to be a graph
  \item (abstraction of "straight line test" about graphs $f:\mathbb{R}\rightarrow\mathbb{R}$)
\end{itemize}

      






%----------------------------------------------------------------------------------------

\end{document}
