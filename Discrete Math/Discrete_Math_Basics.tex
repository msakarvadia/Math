\documentclass{article}
\usepackage[utf8]{inputenc}
\usepackage{graphics}

\usepackage{fancyhdr}
\usepackage{amssymb}

\pagestyle{fancy}
\fancyhf{}
\lfoot{Mansi Sakarvadia}
\lhead{Discrete Mathematics}
\rfoot{Page \thepage}

\begin{document}

\begin{titlepage} % Suppresses headers and footers on the title page

	\centering % Centre everything on the title page
	
	\scshape % Use small caps for all text on the title page
	
	\vspace*{\baselineskip} % White space at the top of the page
	
	%------------------------------------------------
	%	Title
	%------------------------------------------------
	
	\rule{\textwidth}{1.6pt}\vspace*{-\baselineskip}\vspace*{2pt} % Thick horizontal rule
	\rule{\textwidth}{0.4pt} % Thin horizontal rule
	
	\vspace{0.75\baselineskip} % Whitespace above the title
	
	{\LARGE DISCRETE MATHEMATICS\\ MATH 381\\} % Title
	
	\vspace{0.75\baselineskip} % Whitespace below the title
	
	\rule{\textwidth}{0.4pt}\vspace*{-\baselineskip}\vspace{3.2pt} % Thin horizontal rule
	\rule{\textwidth}{1.6pt} % Thick horizontal rule
	
	\vspace{2\baselineskip} % Whitespace after the title block
	
	%------------------------------------------------
	%	Subtitle
	%------------------------------------------------
	
	Basic concepts and examples explaining the fundamentals of Discrete Mathematics. % Subtitle or further description
	
	\vspace*{3\baselineskip} % Whitespace under the subtitle
	
	%------------------------------------------------
	%	Editor(s)
	%------------------------------------------------
	
	By
	
	\vspace{0.5\baselineskip} % Whitespace before the editors
	
	{\scshape\Large Mansi Sakarvadia} % Editor list
	
	\vspace{0.5\baselineskip} % Whitespace below the editor list
	
	\textit{The University of North Carolina \\ Chapel Hill} % Editor affiliation
	
	
\end{titlepage}

\tableofcontents

\section{February 17, 2020}
   
Recall:
\begin{itemize}
  \item Functions f: A\textrightarrow  B
  \item Image $Im(f) = f(a)$
\end{itemize}

\noindent If f(a) = b, say "a is a preimage of b"
\begin{itemize}
  \item $Gr(f) = \{(a,b) | f(a)=b\}\subseteq AxB$
  \item $Gr(f) = \{(a,f(a)) | a \in A\}$
\end{itemize}

\noindent Graph $Gr(f)$ is a relation between A and B\\
\\
\underline{Which binary relations (subsets of AxB) are graphs of functions?}
\begin{itemize}
  \item A subset $s\subseteq AxB$ is the graph of a function if for every element $a \in A$, there is a unique element $b \in B$ such that $(a,b) \in S$.
  \item \underline{key} Can't have $(a, b_1)$ and $(a, b_2)$ $\in$ S where $b_1 != b_2$ and expect S to be a graph
  \item (abstraction of "straight line test" about graphs $f:\mathbb{R}\rightarrow\mathbb{R}$)
\end{itemize}

\subsection{Restriction of Domain}

\indent suppose $f:A \rightarrow B$\\
\indent Consider $A'\subseteq A$\\ \\
\indent DEFINITION: the restriction of f to A' is $f|_{A'}: A' \rightarrow B$ defined by $(f|_{A'})=f(a), a \in A$\\

\indent KEY POINT: What does it mean for 2 functions $f:A \rightarrow B$ and $g:C \rightarrow D$ to be equal?
\begin{itemize}
\item NEED: A=C, B=D, and $f(a)=g(a) a \in A $\\
\end{itemize}

\subsection{Restriction of Codomain}
If B' is a set with $Im(f) \subseteq B' \subseteq B$, then we consider:

\indent $f':A \rightarrow B'$\\
\indent defined by  $f'(a) = f(a) for all a \in A$\\ \\
\indent EX. $Im(f|_{A'}) = f(A')$\\
\begin{itemize}
\item $Im(f|_{A'}) \leftarrow$ image of the restriction of f to A'
\item $f(A') \leftarrow$ image of the subset $A' \subseteq A$ under f\textbf{•} 
\end{itemize}

\subsection{Arithmetic of functions}
\begin{itemize}
\item A function is called \underline{real-valued} if its codomain is $\in \mathbb{R}$
\item A function is called \underline{inter-valued} if its codomain is $\in \mathbb{Z}$
\item DEFINITION: Suppose that $f_{1}$ and $f_{2}$ are two real-valued function both w/ domain A. Then we have $f_{1} + f_{2}$ and $f_{1}f_{2}$ (the sum and product), two real-valued functions on A, defined by:
\begin{itemize}
\item $f_{1} + f_{2}(x) = f_{1}(x) + f_{2}(x)$
\item $f_{1}f_{2}(x) = f_{1}(x) * f_{2}(x)$
\item $f_{1},f_{2}:\mathbb{R} \rightarrow \mathbb{R}$ given by $f_{1}(x)=x^2, f_{2}(x)=x-x^2$
\begin{itemize}
\item Then $f_{1}+f_{2}, f_{1}*f_{2}:\mathbb{R} \rightarrow \mathbb{R}$ are defined by:\\
			$f_{1} + f_{2}(x) = f_{1}(x) + f_{2}(x)=x $\\
			$f_{1}f_{2}(x) = f_{1}(x) * f_{2}(x)=x^3-x^4$
\item Note: $f_{1} + f_{2}(x) = \iota_{\mathbb{R}}$ (hard to read symbol is iota subscript real-number symbol)
\end{itemize}
\end{itemize}
\end{itemize}

\subsection{Injective, Surjective, Bijective}
These are properties that a function may or may not have.\\
Consider $f: A \rightarrow B$\\

\underline{Injective}\\

DEFN: Say that f is injective if for any $x,y \in A$, $f(x)=f(y) \rightarrow x=y$\\
\begin{itemize}
\item Equivalent Statements:
\begin{itemize}
\item Whenever x!=y belong to A be must have f(x)!=f(y)
\begin{itemize}
\item contra-positive, "Distinct points of A map to distinct values of B"
\end{itemize}
\item For every b in B, there is at most one a in A which f(a)= b.
\begin{itemize}
\item "every point of B has at most one preimage."
\end{itemize}
\item Eg. $f:\mathbb{Z} \rightarrow \mathbb{Z}$ defined by $f(n) = n^2$
\begin{itemize}
\item \underline{not injective} because f(-1) = f(1).
\item $f|_{\mathbb{Z^{> 0}}}$ is injective (function where domain is integer's greater than zero)
\end{itemize}
\item How to prove?
\begin{itemize}
\item Criteria for injectivity\\
		Suppose $f: A \rightarrow B$ where B, A are subsets of real numbers
\begin{itemize}
\item f is increasing if $x<y \rightarrow f(x) <= f(y)$
\item f is strictly increasing if $x<y \rightarrow f(x) < f(y)$
\item f is decreasing if $x<y \rightarrow f(x) >= f(y)$
\item f is strictly decreasing if $x<y \rightarrow f(x) > f(y)$
\end{itemize}
\item Proposition: A function that is strictly increasing or decreasing is injective
\end{itemize}
\end{itemize}
\end{itemize}









%----------------------------------------------------------------------------------------

\end{document}
